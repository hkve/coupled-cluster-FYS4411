\subsection{Mathematical framework and notation}
In the following we will use the occupation representation, making use of creation $\crta{p}$ and annihilation $\ania{p}$ operators. As a shorthand, we will write $\crt{p} \equiv \crta{p}$ and $\ani{p} \equiv \ania{p}$ when no confusion can be made. $N$ represents the number of occupied states while $L$ the total number of states in our calculations. The indicies $p,q,\ldots$ are reserved for the $L$ general states, the $N$ occupied states are indexed by $i,j,\ldots$, while the $L-N$ unoccupied (virtual) states by $a,b,\ldots$ indicies.

Since we are treating fermionic systems, the canonical anticommutation relations are used

\begin{align*}
    \set{\crt{p},\crt{q}} = \set{\ani{p},\ani{q}} = 0 \hspace{20px} \set{\crt{p},\ani{q}} = \delta_{pq}.
\end{align*}
One and two body matrix elements are calculated using a computational basis, with explicit expressions for one body Hamiltonians $h(\vec{x})$ and two body interaction $v(\vec{x}, \vec{x}')$ here presented in position space

\begin{align*}
    \bra{p}\hat{h}\ket{q} &= \int \dd{\vec{x}} \psi_p^*(x) \hat{h}(\vec{x}) \psi_q(x) \\
    \elm{pq}{rs} &= \int \dd{\vec{x}}\dd{\vec{x}'} \psi_p^*(\vec{x})\psi_q^*(\vec{x}') \hat{v}(\vec{x},\vec{x}') \psi_r(\vec{x})\psi_s(\vec{x}')
\end{align*}
with $\psi_p$ being a single particle wave function, often chosen to be the eigenfunction of $\hat{h}$. Note that $p$ also contain the spin quantum number, meaning that $\dd \vec{x}$ implicitly contains a spin component. If $\hat{h}$ or $\hat{v}$ is spin \textit{independent}, this simply reduces to Kronecker deltas for the spin component. It is often convenient to use antisymmetrized matrix elements, defined as 
\begin{align*}
    \elmAS{pq}{rs} = \elm{pq}{rs} - \elm{pq}{sr}
\end{align*}
The shorthands $\hshort{p}{q} \equiv \bra{p} \hat{h} \ket{q}$, $\elmshort{pq}{rs} \equiv \elm{pq}{rs}$ and $\elmASshort{pq}{rs} \equiv \elmAS{pq}{rs}$ will often be used. Using this formulation, a general two-body operator can be constructed

\begin{align}
    \ham = \sum_{pq} \hshort{p}{q}\crt{p}\ani{q} + \frac{1}{4} \sum_{pqrs} \elmASshort{pq}{rs}\crt{p}\crt{q}\ani{s}\ani{r} \label[eq]{eq:theo:second_quant_ham}
\end{align}
The simplest ground state ansatz
\begin{align*}
    \ketansatz = \crt{i}\crt{j}\ldots \ket{0},
\end{align*}
can be evaluated to calculate the simplest energy estimate using \text{Wicks Theorem} \citep{molinariNotesWickTheorem2017} 
\begin{align}
    \braansatz \ham \ketansatz = \sum_{i} \hshort{i}{i} + \frac{1}{2}\sum_{ij} \elmASshort{ij}{ij} \equiv \Eref, \label[eq]{eq:theo:e0ref}
\end{align}
named the \textit{reference energy}. Commonly wicks theorem is applied to \cref{eq:theo:second_quant_ham} to pick out the \cref{eq:theo:e0ref} contribution, defining the \textit{normal ordered} Hamiltonian
\begin{align*}
    \ham &= \hamno + \Eref = \fockno + \interno + \Eref
\end{align*}
where $\fockno$ and $\interno$ is the normal ordered \textit{Fock operator} and two body interaction respectively.
\begin{align}
    \fockno &= \sum_{pq} \fshort{p}{q} \set{\crt{p} \ani{q}}, \label[eq]{eq:theo:fock_no} \\
    \interno &= \frac{1}{4}\sum_{pqrs} \elmASshort{pq}{rs} \set{\crt{p}\crt{q}\ani{s}\ani{r}} \label[eq]{eq:theo:interaction_no}
\end{align}
The operators inside the curly brackets denotes \textit{normal ordering}. In constructing the Fock operator, the matrix elements $\fshort{p}{q}$ are given as 
\begin{align*}
    \fshort{p}{q} = \hshort{p}{q} + \sum_i \elmASshort{pi}{qi}
\end{align*}
One major reason for doing this is the applicability of the \textit{Generalized Wicks Theorem}, such that we only need to consider contractions between different normal ordered strings \citep{ferialdiGeneralWickTheorem2021}.

\begin{align}
    \hat{H}\ket{\Psi} = E \ket{\Psi} \label[eq]{eq:theo:schroding_equation}
\end{align}

\subsection{Hartree-Fock}
The Hartree-Fock method is one of the cheapest and most commonly applicated many-body methods. Using the reference energy equation \cref{eq:theo:e0ref}, we perform a basis change to the Hartree-Fock basis, based on minimizing the ansatz expectation value. Using Greek letters to index the computational basis $\alpha, \beta \ldots$, going over all states $L$ we can change to the Hartree-Fock basis using

\begin{align*}
    \ket{p} = \sum_{\alpha} C_{\alpha p}\ket{\alpha}
\end{align*}
where $C_{\alpha p}$ are the basis coefficients. Assuming the 

We are however not free to choose an arbitrary transformation, since we re 

\begin{align}
    \mathcal{L} = \braansatz \ham \ketansatz + \sum_i \epsilon_{i\alpha} (\delta_{ij} - \ip{i}{j})
\end{align}

\subsection{Coupled Cluster}
The exact solution $\ket{\Psi}$ is approximated by an exponential ansatz $\ketCC$ 
\begin{align}
    \ket{\Psi} \approx \ketCC \equiv \exp{\cluster{}} \ketansatz. \label[eq]{eq:theo:exponential_ansatz}
\end{align}
The operators $\cluster{} = \cluster{1} + \cluster{2} + \ldots$ acting on the ground state ansatz $\ketansatz$ are the so-called \textit{cluster operators} defined as
\begin{align}
    \cluster{m} = \frac{1}{(m!)^2} \sum_{\substack{ab\ldots\\ij\ldots}} \amplitude{ab\ldots}{ij\ldots} \set{\crt{a}\ani{i} \crt{b} \ani{j}\ldots} \label[eq]{eq:theo:cluster_operators_general}
\end{align}
where $m \leq N$. The scalars $\amplitude{ab\ldots}{ij\ldots}$ are unknown expansion coefficients called \textit{amplitudes}, which we need to solve for. All the creation and annihilation operators of \cref{eq:theo:cluster_operators_general} anticommute, giving the restriction that

\begin{align}
    \amplitude{\hat{P}(ab\ldots)}{\hat{P}'(ij\ldots)} = (-1)^{\sigma(\hat{P})+\sigma(\hat{P}')} \amplitude{ab\ldots}{ij\ldots}. \label[eq]{eq:theo:amplitude_permutation_symmetry}
\end{align}
Here $P$ and $P'$ permutes $\sigma(P)$ and $\sigma(P')$ indices respectively. This is the reason for the prefactor of \cref{eq:theo:cluster_operators_general}, since we have $m!$ ways to independently permute particle and hole indices. Instead of having $(L-N)^m N^m$ independent unknowns, we reduce this number by a factor of $(m!)^{2}$. 

\subsection{Doubles truncation}
Considering $N$ cluster operators in the exponential ansatz of \cref{eq:theo:exponential_ansatz} is not computationally feasible for realistic systems. The common practice is to include one or more $\cluster{m}$ operators, making a truncation on $\ketCC$ as well. In the following we will include only the double excitation operator $\cluster{2}$, know as the CCD approximation. This gives us 
\begin{align}
    \ket{\Psi} &\approx \ketCC \approx \ketCCD \equiv \exp{\cluster{2}} \ketansatz \label[eq]{eq:theo:CCD_exponential_ansatz}, \\
    \cluster{2} &= \frac{1}{4}\sum_{abij} \amplitude{ab}{ij}\set{ \crt{a} \ani{i} \crt{b}\ani{j}} \label[eq]{eq:theo:CCD_cluster},
\end{align}
with the four-fold amplitude permutation symmetry \footnote{For double amplitudes, the index permutation symmetry is equal to that of antisymmetrized two-body matrix elements $\elmAS{pq}{rs}$.},
\begin{align}
    \amplitude{ab}{ij} = - \amplitude{ba}{ij} = - \amplitude{ab}{ji} = \amplitude{ba}{ji}. \label[eq]{eq:theo:CCD_amplitude_symmetry}
\end{align}
Incorporating the CCD approximation in the Schr\"odinger equation (\cref{eq:theo:schroding_equation}), we see that
\begin{align}
    \ham \exp{\cluster{2}} \ketansatz &= E \exp{\cluster{2}} \nonumber \ketansatz,  \\
    \hamno \exp{\cluster{2}} \ketansatz &= \DECCD \exp{\cluster{2}} \ketansatz, \label[eq]{eq:theo:schroding_equation_CCD}
\end{align}
where $\DECCD = E-\Eref$. Expanding both sides and taking the inner product with $\braansatz$, we in principle get an equation for the energy. However, this approach is not amenable to practical computer implementation \citep{bartlettCoupledClusterMethodsMolecular1984} since the amplitude equation will be coupled with the energy equation. Therefor, we rather apply a similarity transform to \cref{eq:theo:schroding_equation_CCD} by multiplying by the inverse of $\exp{\cluster{2}}$,
\begin{align}
    \exp{-\cluster{2}} \hamno \exp{-\cluster{2}} \ketansatz &= \DECCD \ketansatz \nonumber \\
    \simham \ketansatz &= \DECCD \ketansatz \label[eq]{eq:theo:ccd_eigenvalueproblem}
\end{align}
where $\simham = \exp{-\cluster{2}} \hamno \exp{-\cluster{2}}$ is the similarity transformed Hamiltonian. Using this reformulated eigenvalue problem, we can perform the inner product with different states to calculate both $\DECCD$ and $\amplitude{ab}{ij}$. Considering $\braansatz$ we get

\begin{align}
    \braansatz \simham \ketansatz = \DECCD, \label[eq]{eq:theo:energy_equation_sim_transformed}
\end{align}
named the \textit{energy equation}. Considering excited states, we arrive at the \textit{amplitude equations}
\begin{align}
    \braexed{ab\ldots}{ij\ldots} \simham \ketansatz \label[eq]{eq:theo:amplitude_equations},
\end{align}
used for finding the unknown amplitudes $\amplitude{ab}{ij}$. To find explicit expressions for \cref{eq:theo:energy_equation_sim_transformed} and \cref{eq:theo:amplitude_equations}, we expand $\simham$ using the Hausdorff expansion
\begin{align*}
    \simham &= \hamno + \commutator{\hamno}{\cluster{2}} + \frac{1}{2!}\commutator{\commutator{\hamno}{\cluster{2}}}{\cluster{2}}.
\end{align*}
The truncation at the two-fold commutator comes from the fact that we have a two-body interaction. When evaluated with a doubly excited state, at least one creation or annihilation operator from each of the cluster operators has to be contracted with $\hamno$. Therefor having eight creation and annihilation operators in two $\cluster{2}$, four of them can be contracted with the four from $\hamno$, while the other four with the operators from $\braexed{ab}{ij}$. This gives the CCD amplitude equation calculated from 

\begin{align}
    \braexed{ab}{ij}\simham\ketansatz = 0 \label[eq]{eq:theo:amplitude_equation_CCD}
\end{align}
To make practical use of \cref{eq:theo:energy_equation_sim_transformed} and \cref{eq:theo:amplitude_equation_CCD}