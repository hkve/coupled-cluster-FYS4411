\subsection{Computational Considerations}
The methods used to solve both the HF and CCD equations will be outlined in the following section. 

\subsubsection{Hartree-Fock}
Solving \cref{eq:theo:hartreefock_eigenvalueproblem} is as stated simply an eigenvalue problem, with $\epsilon$ and $C$ as eigenvalues and vectors respectively. However, we see from \cref{eq:theo:hartree_fock_matrix_and_density} that the construction of $\hamhf{\alpha\beta}$ requires the density matrix $\rho$ which again is dependent on the coefficients. To find the correct minimization coefficients an iterative scheme was applied. The initial guess was always chosen to be $C^{(0)}_{ii} = 1$ \footnote{Note that this is over occupied states only}. As a stopping criterion, the sum of Lagrange multipliers per occupied states between iterations $\Delta$ was used. The iterations stop when $\Delta$ is below a predetermined tolerance $\Delta_0$. This approach is outlined in \cref{algo:met:hartreefock}.

\begin{algorithm}[H]
    \begin{algorithmic}
        \State $C^{(0)}_{ii} \gets 1$
        
        \While{$\Delta > \Delta_0$}
        \State $\hamhf{} \gets f(C^{(n-1)})$ \cref{eq:theo:hartree_fock_matrix_and_density}
        \State $C^{(n)} \gets \text{Eigenvectors}(\hamhf{})$
        \State $\epsilon^{(n)} \gets \text{Eigenvalues}(\hamhf{})$
        \State $\Delta \gets \sum |\epsilon^{(n)}-\epsilon^{(n-1)}|/N$ 
        \EndWhile
    \end{algorithmic}
    \caption{Outline of Hartree-Fock iterative scheme.} \label[algo]{algo:met:hartreefock}
\end{algorithm}


\subsubsection{Coupled Cluster}
The approach for solving the amplitudes in \cref{eq:theo:amplitude_equation_CCD} follow much the same methodology as the HF iterative scheme. Fixed point iterations are a common solution to solving non-linear sets of coupled equations, and is relatively easy to formulate.

Going back to the amplitude equation \cref{eq:theo:Amplitude_CCD_explicit} we separate the diagonal and off-diagonal part of $f_{pq}$ as $f_{p}^D$ and $f_{pq}^O$ respectively, giving

\begin{align*}
    f_{pq} = \delta_{pq} f_{pp} + (1-\delta_{pq})f_{pq} = f_{p}^D + f_{pq}^O.
\end{align*}
Considering the $\permoperator{ab}$ term with the Fock matrix
\begin{align*}
    \permoperator{ab} \sum_{c} f_{bc} \amplitude{ac}{ij} = \sum_{c} f_{bc}\amplitude{ac}{ij} - \sum_{c}f_{ac}\amplitude{bc}{ij},
\end{align*}
we can insert the diagonal and off-diagonal partition of the Fock matrix giving

\begin{align*}
    \sum_{c} t_{bc} \amplitude{ac}{ij} &= \sum_{c} \delta_{bc}f_{bb}\amplitude{ac}{ij} + (1-\delta_{bc})f_{bc}\amplitude{bc}{ij} \\
    &= f_{b}^D \amplitude{ab}{ij} + \sum_{c}f_{bc}^O \amplitude{ac}{ij}
\end{align*}
Similarly 
\begin{align*}
    \sum_{c} f_{ac}\amplitude{bc}{ij} = -f_a^D \amplitude{ab}{ij} + \sum_{c} f_{ac}^O \amplitude{bc}{ij}
\end{align*}
Meaning that the $\permoperator{ab}$ Fock matrix term can be expressed as the sum over the off-diagonal part, in addition to a no-sum $f_a^D + f_b^D$ term.
\begin{align*}
    \permoperator{ab} \sum_{c} f_{bc}\amplitude{ac}{ij} = (f_a^D + f_b^D)\amplitude{ab}{ij} + \permoperator{ab} \sum_{c} f_{bc}\amplitude{ac}{ij}
\end{align*}
Performing the same decomposition for the $\permoperatorshort{ij}$ yields the same decomposition. If we go back to \cref{eq:theo:Amplitude_CCD_explicit} and call every term except the pure diagonal term $R^{ab}_{ij}(t)$ \footnote{Note that this is a function of the amplitudes \textit{resulting} in the four-index term after the sums has been performed.} we get 
\begin{align*}
    \mathcal{F}^{ab}_{ij}\amplitude{ab}{ij} + R^{ab}_{ij}(t) &= 0, \\ 
    \amplitude{ab}{ij} &= -\frac{R^{ab}_{ij}(t)}{\mathcal{F}^{ab}_{ij}},
\end{align*}
where we have defined 
\begin{align*}
    \mathcal{F}^{ab}_{ij} = f_a^D + f_b^D -f_i^D - f_j^D.
\end{align*}
Note that $\mathcal{F}^{ab}_{ij}$ is iteration independent and can be precomputed. This gives a function suitable for fixed point iterations. It is also possible to calculate the new amplitudes using some percentage of the old amplitudes, \textit{slowly mixing} in the new iterations. Parameterized by the mixing parameter $p \in [0,1)$, we update the amplitudes following  

\begin{align}
    \amplitude{ab,(n)}{ij} = p\amplitude{ab,(n-1)}{ij} + (1-p)\frac{R^{ab}_{ij}(t^{(n-1)})}{\mathcal{F}^{ab}_{ij}} \label[eq]{eq:met:amplitudes_update_rule}
\end{align}
Convergence is determined by finding stable solutions to $\amplitude{ab}{ij}$ thus we evaluate the CCD contribution \cref{eq:theo:Energy_CCD_explicit} each iteration and stop when the difference between iterations is lower than some predetermined tolerance. The procedure is outlined in \cref{algo:met:ccd}

\begin{algorithm}[H]
    \begin{algorithmic}
        \State $\amplitude{ab(0)}{ij} \gets 0$
        
        \While{$\Delta > \Delta_0$}
        \State $\amplitude{ab,(n)}{ij} \gets p\amplitude{ab,(n-1)}{ij} + (1-p)\frac{R^{ab}_{ij}(t^{(n-1)})}{\mathcal{F}^{ab}_{ij}} $ \cref{eq:met:amplitudes_update_rule}
        \State $E^{(n)} \gets \sum \frac{1}{4} \elmASshort{ij}{ab}\amplitude{ab,(n-1)}{ij}$ \cref{eq:theo:Energy_CCD_explicit}
        \State $\Delta \gets |E^{(n)}-E^{(n-1)}|$ 
        \EndWhile
    \end{algorithmic}
    \caption{Outline of CCD iterative scheme.} \label[algo]{algo:met:ccd}
\end{algorithm}
The choice of all zeros for $t^{(0)}$ is somewhat arbitrary, and different choices can be made constrained to the symmetry requirements of \cref{eq:theo:amplitude_permutation_symmetry}. Since we have chosen all zeros, the first iteration will always yield

\begin{align*}
    \amplitude{ab,(1)}{ij} = \frac{\elmASshort{ab}{ij}}{\mathcal{F}^{ab}_{ij}}
\end{align*}
which gives an energy equal to the many-body perturbation theory to the second order (MBPT2)
\begin{align*}
    \DECCD^{(1)} = \sum_{\substack{ab\\ij}} \frac{|\elmASshort{ab}{ij}|^2}{f_a^D + f_b^D -f_i^D - f_j^D} = \Delta E_{\text{MBPT2}}.
\end{align*} 
This displays the perturbative nature of the CC equations. Alternatively one could initialize the amplitudes to give the MBPT2 energy, however due to the miniscule performance enhancements for the systems investigated here, this was not considered necessary.   
\subsubsection{Numerical efficacy}
The iterative amplitude scheme relies heavily on contraction of the rank-4 tensors $\amplitude{pq}{rs}$ and $\elmASshort{pq}{rs}$. Taking the four contraction term with no permutation from \cref{eq:theo:Amplitude_CCD_explicit} remembering the summation convention

\begin{align}
    \frac{1}{4}\elmASshort{kl}{cd}\amplitude{cd}{ij}\amplitude{ab}{kl} \equiv \mathcal{T}^{ab}_{ij}, \label[eq]{eq:met:term_to_be_optimized}
\end{align}
we see that we have two sums over the virtual states contributing a factor $M^2$, while also having a sum over occupied states contributing $N^2$. Since we create an object $\mathcal{T}^{ab}_{ij}$, these contractions have to be performed for $M^2$ virtual and $N^2$ occupied indices, giving this term a total time-complexity of $\mathcal{O}(N^4 M^4)$. Considering a sum over an \textit{intermediate} tensor 
\begin{align*}
    \chi^{ab}_{cd} = \frac{1}{4}\amplitude{ab}{kl}\elmASshort{kl}{cd},
\end{align*}
we latch on a time complexity $\mathcal{O}(M^4 N^2)$. By then again contracting with an amplitude, we recover the original contraction from \cref{eq:met:term_to_be_optimized}
\begin{align*}
    \amplitude{cd}{ij}\chi^{ab}_{cd} = \frac{1}{4} \amplitude{cd}{ij}\amplitude{ab}{kl} \elmASshort{kl}{cd}
\end{align*}
which again has the time complexity $\mathcal{O}(M^4 N^2)$. At the penalty of storing the $M^4$ elements of $\chi^{ab}_{cd}$, we have reduced the time complexity by a factor of $N^2$. We could also include more terms in $\chi^{ab}_{ij}$ such that more than one term of \cref{eq:theo:Amplitude_CCD_explicit} can be computed
\begin{align*}
    \chi^{ab}_{ij} &= \frac{1}{4}\amplitude{ab}{kl}\elmASshort{kl}{cd} + \frac{1}{2} \elmASshort{ab}{cd} \\
    \amplitude{cd}{ij}\chi^{ab}_{cd} &= \frac{1}{4} \amplitude{cd}{ij}\amplitude{ab}{kl} \elmASshort{kl}{cd} + \frac{1}{2}\elmASshort{ab}{cd}\amplitude{cd}{ij}
\end{align*}
which will be faster, but not lower the over scaling below $\mathcal{O}(M^4 N^2)$. For the CCD equations, there are no reuse of intermediates and only the summation order yields better time complexities. Through the tensor contraction functionality of \verb|NumPy| \citep{vanderwaltNumPyArrayStructure2011}, optimal paths can be calculated before the summation is performed. Therefor the potential optimizations of this approach was deemed neglectable. 

There are however more possibilities for optimization which was not considered here. Based on symmetries of the two-body interaction, only matrix elements which we a priori know will not necessarily be zero can be considered by storing matrix elements and amplitudes in different \textit{symmetry channels}. Additionally, with some index mapping, matrix storage can used allowing for full \verb|BLAS| functionality \citep{blackford2002updated}.

\subsection{Spin restriction}
Both the HF and CCD frameworks presented here treats matrix elements including spin. If the Hamiltonian is spin-independent, both storing and summation over matrix elements can be significantly reduced. Separating relevant quantum numbers and spin, explicitly $p = (P,\sigma_P)$ with $P$ as relevant numbers and $\sigma_P$ as spin, we see that a general matrix element
\begin{align*}
    \elm{pq}{rs} = \elm{PQ}{RS} \delta_{\sigma_P \sigma_R} \delta_{\sigma_Q \sigma_S},
\end{align*}
only gives non-zero contributions when both the spins of $p,r$ and $q,s$ align. This reduces the number of matrix elements by a constant factor of 16, improving both storage and computation time. To make use of this however, we must rewrite our equations to use normal matrix elements, instead of antisymmetrized elements. 
\subsubsection{Hartree-Fock}
Taking the reference energy term from the HF Lagrangian, we can explicitly sum out the spin from the one and two body terms. Starting with the diagonal one body term we have

\begin{align*}
    \sum_i \hshort{i}{i} = \sum_{I\sigma_I} \hshort{I}{I}\delta_{II} = 2\sum_I \hshort{I}{I}.
\end{align*}
By expanding the antisymmetrized matrix elements, the two body term follows similarly
\begin{align*}
    \frac{1}{2}\sum_{ij} \elmASshort{ij}{ij} &= \frac{1}{2}\sum_{ij} \elmshort{ij}{ij} - \elmshort{ij}{ji} \\
    &= \frac{1}{2}\sum_{IJ\sigma_I \sigma_J} \elmshort{IJ}{IJ}\delta_{\sigma_I \sigma_I}\delta_{\sigma_J \sigma_J} - \elmshort{IJ}{JI}\delta_{\sigma_I \sigma_J}\delta_{\sigma_I \sigma_J} \\
    &= \sum_{IJ} \pclosed{2\elmshort{IJ}{IJ} -  \elmshort{IJ}{JI}}.
\end{align*}
The reference energy term \cref{eq:theo:e0ref} can be expressed with spin summed out as
\begin{align*}
    \sum_i \hshort{i}{i} + \frac{1}{2}\sum_{ij} \elmASshort{ij}{ij} = 2\sum_I \hshort{I}{I} + \sum_{IJ} \pclosed{2\elmshort{IJ}{IJ}-\elmshort{IJ}{JI}}.
\end{align*}
Now the same expansion in the computational basis can be performed such as in \cref{sec:theo:hartreefock} and by optimizing wrt. to the coefficients $C_{AI}$ with the orthonormality constraint, we get a similar HF matrix
\begin{align*}
    \hamhf{AB} = h_{AB} + \sum_{G D} \rho_{GD} \elmshort{A G}{B D} - \frac{1}{2}\sum_{G D} \rho_{GD}\elmshort{AG}{DB}
\end{align*}
where we now have defined the density matrix with a factor two, reflecting the double occupancy of each (spatial) state.
\begin{align*}
    \rho_{GD} = 2\sum_i C^*_{G I} C_{D I}
\end{align*}
The same iterative scheme from \cref{algo:met:hartreefock} was used, with the modified HF matrix and energy evaluation.

\subsubsection{Coupled Cluster}
Motivated by the HF approach, we expand the antisymmetrized matrix elements. Since we additionally have antisymmetric amplitudes, we also expand these. The CCD energy from \cref{eq:theo:Energy_CCD_explicit} then becomes 
\begin{align*}
    \DECCD  &= \frac{1}{4} \sum_{\substack{ab\\ij}} \elmASshort{ij}{ab}\amplitude{ab}{ij} = \frac{1}{8}\sum_{\substack{ab\\ij}} (\elmshort{ij}{ab} - \elmshort{ij}{ba})(\amplitude{ab}{ij}- \amplitude{ba}{ij})
\end{align*}
We now defined new amplitudes where the antisymmetry is gone, but still the symmetry of the normal two body elements \cref{eq:theo:matrix_elements_symmetry} is present.
\begin{align}
    \amplituderestrict{ab}{ij} = \amplituderestrict{ba}{ji} = \amplituderestrict{AB}{IJ}\delta_{\sigma_{A}\sigma_{I}}\delta_{\sigma_{B}\sigma_{J}} = \amplituderestrict{BA}{JI}\delta_{\sigma_{B}\sigma_{J}}\delta_{\sigma_{A}\sigma_{I}} \label[eq]{eq:met:amplitude_restricted_symmetry}
\end{align}
As shown in detail in \cref{sec:app:ccd_spinrestriction}, the CCD energy can then be rewritten as
\begin{align*}
    \DECCD  &= \frac{1}{2}\sum_{\substack{ab\\ij}} \elmshort{ij}{ab} \amplituderestrict{ab}{ij} - \elmshort{ij}{ba} \amplituderestrict{ab}{ij} \\
    &= \sum_{\substack{AB\\IJ}} \pclosed{2 \elmshort{IJ}{AB} - \elmshort{IJ}{BA}} \amplituderestrict{AB}{IJ}
\end{align*}
Similar approaches can be used to sum the spin out of \cref{eq:theo:amplitude_equation_CCD}, which is a drudging and lengthy task. Here the results from  \citep{shavitt_bartlett_2009} is presented. The result has also been calculated using \verb|Drudge|, present in the code base. Einstein summation convention is assumed

\begin{align}
    \begin{split}\label[eq]{eq:met:amplitude_equation_expanded_resricted}
        &\elmshort{AB}{IJ} + \permoperator{(IA)(JB)} \Big[ f_{BC}\amplituderestrict{AC}{IJ} -f_{KJ}\amplituderestrict{AB}{IK} \\
        &+ \frac{1}{2}\elmshort{AB}{CD}\amplituderestrict{CD}{IJ} + \frac{1}{2}\elmshort{KL}{IJ}\amplituderestrict{AB}{KL} + 2 \elmshort{KB}{CJ}\amplituderestrict{AC}{KI} \\
        &-\elmshort{KB}{CJ}\amplituderestrict{AC}{KI} - \elmshort{KB}{IC}\amplituderestrict{AC}{KJ} - \elmshort{KB}{JC}\amplituderestrict{AC}{IK} \\
        &+\elmshort{KL}{CD}\Big( \frac{1}{2}\amplituderestrict{CD}{IJ}\amplituderestrict{AB}{KL} + 2\amplituderestrict{AC}{IK}\amplituderestrict{DB}{LJ}-2\amplituderestrict{AC}{IK}\amplituderestrict{DB}{JL} \\
        &+\frac{1}{2}\amplituderestrict{CA}{IK}\amplituderestrict{BD}{LJ}-\amplituderestrict{AD}{IK}\amplituderestrict{CB}{LJ}+\amplituderestrict{AD}{KI}\amplituderestrict{CB}{LJ} \\
        &+\frac{1}{2}\amplituderestrict{CB}{IL}\amplituderestrict{AD}{KJ} -2\amplituderestrict{CD}{KI}\amplituderestrict{AB}{LJ}+ \amplituderestrict{CD}{IK}\amplituderestrict{AB}{LJ}\\
        &-2\amplituderestrict{CA}{KL}\amplituderestrict{DB}{IJ}+\amplituderestrict{AC}{KL}\amplituderestrict{DB}{IJ}\Big)
        \Big] = 0
    \end{split}
\end{align}
with
\begin{align*}
    \permoperator{(IA)(JB)} = 1 + \permoperatorshort{IJ}\permoperatorshort{AB}
\end{align*}

\subsection{Quantum Mechanical System}

\subsubsection{Helium and Beryllium}
The initial testing during development of the CCD and RCCD implementations was performed using Hydrogen wave functions. As a choice of basis sets, these functions are `physically motivated` in the sense of being solutions to the one body electron case. The well-know relevant quantum numbers determining the form of the spatial wave functions are $n$ as the principal quantum number, with $l$ and $m$ as orbital angular momentum and projection respectively. Due to the spherically symmetric potential, the wave function $\psi_{nlm}$ can be separated in a radial function $R_{nl}(r)$ and a spherical harmonic $\sphericalharmonic{m}{l}$,  

\begin{align}
    \psi_{nlm}(r,\theta,\phi) = R_{nl}(r) \sphericalharmonic{m}{l}(\theta, \phi). \label[eq]{eq:met:hydrogen_wf_definition}
\end{align}
The radial part $R_{nl}(r)$ has the form

\begin{align*}
    R_{nl}(r) &= A_{nl} e^{-r/na_0} \pclosed{\frac{2r}{na_0}}^l \bclosed{\associatedlaguerre{2l+1}{n-l-1}(2r/na_0)} \\
    A_{nl} &= \sqrt{ \pclosed{\frac{2}{na_0}}^3 \frac{(n-l-1)!}{2n(n+l)!} }
\end{align*}
With $L$ as the associated Laguerre polynomials and $a_0$ the Bohr radius. For $l > 0$ the Coulomb interaction integral of \cref{eq:met:hydrogen_wf_definition} can not be easily evaluated due to $\sphericalharmonic{m}{l}$ having a non-trivial $\theta$ and $\phi$ dependence. Therefor for simplicity we only consider $s$ orbital ($l=0$ states), when calculating $\elm{pq}{rs}$. This is briefly sketched in \cref{sec:app:hydrogen_coulomb_integrals}. The one-body term $\hshort{p}{q}$ are diagonal with 

\begin{align}
    h_{nm} = -\frac{Z}{2n^2}\delta_{nm}
\end{align}
We will be interested in calculating the ground state energy for Helium and Beryllium, with two and four electrons respectively. In addition to Hartree-Fock and CCD calculations, a comparison to configuration interaction using singles (CIS) will be performed for both Helium and Beryllium. The results will also be benchmarked against the famous work done by Egil A. Hylleraas \citep{hylleraasNumerischeBerechnung2STerme1930}.

\subsubsection{Two-Dimensional Harmonic Oscillator}
To test the HF and CCD implementations on larger basis sets, the two-dimensional harmonic oscillator was chosen. $N$ electrons are confined in a potential characterized by the oscillation frequency $\omega$, with a repulsive Coulomb term

\begin{align}
    H = \sum_{i=1}^N \pclosed{-\frac{1}{2}\nabla_i^2 + \frac{1}{2}\omega^2 r_i^2} + \sum_{i < j} \frac{1}{r_{ij}} \label[eq]{eq:met:2dho_hamiltonian}
\end{align}
where $r_{ij} = |\vec{r}_i - \vec{r}_j|$. The natural single particle basis for this problem is the solutions to the non-interactive harmonic oscillator case. Characterized by two quantum numbers $n_x$ and $n_y$, the position space wave function in Cartesian coordinates are expressed as 

\begin{align*}
    \psi_{n_x n_y} (x,y) &= A_{n_x n_y} H_{n_{x}} (\sqrt{\omega}x) H_{n_{y}} (\sqrt{\omega}y) e^{-\omega (x^2 + y^2)/2} \\
    A_{n_x n_y} &= \sqrt{\frac{\omega}{\pi 2^{(n_x+n_y)}n_x ! n_y !}}
\end{align*}
The one-body Hamiltonian is diagonal, with the well known energies

\begin{align}
    E_{n_x,n_y} = \hbar \omega (n_x + n_y + 1) \label[eq]{eq:met:ho_sp}
\end{align}
for two dimensions. For the Coulomb integral from \cref{eq:met:2dho_hamiltonian} a more nuanced consideration is in order. The simplest approach is to solve the Coulomb integrals numerically, however even though we just work in two dimensions, the integral will be two-dimensional with a quite complex integrand. For larger basis sets where $L \sim 100$, this approach can not be performed without any clever performance tricks. Luckily this problem has been solved analytically in spherical coordinates \citep{anisimovasEnergySpectraFewelectron1998}, giving a much cheaper way to incorporate the Coulomb integrals. The implementation co-authored by Ø. Schøyen has been used \footnote{Implementation is present at \url{https://github.com/HyQD/quantum-systems/tree/master}}.

We will consider closed shell systems, that is the particle number $N$ will always correspond to all sets of $(n_x, n_y)$ which gives the same single particle energy from \cref{eq:met:ho_sp} and every energy below this. These shell closures are tabulated in \cref{tab:met:ho_shell_numbers}.

\begin{table}[H]
    \centering
    \begin{tabular}{c|ccc}
    $R$ &  $(n_x,n_y)$ & $d$ & $N$ \\
    \hline
    1 & $(0,0)$ & 2 & 2\\
    2 & $(1,0)_1$ & 4 & 6\\
    3 & $(2,0)_1, (1,1)$ & 6 & 12\\
    4 & $(3,0)_1,(2,1)_1 $ & 8 & 20 \\
    5 & $(4,0)_1, (3,1)_1, (2,2)$ & 10 & 30 \\
    6 & $(5,0)_1, (4,1)_1, (3,2)_1$ & 12 & 42 \\
    7 & $(6,0)_1, (5,1)_1, (4,2)_1,(3,3)$ & 14 & 56 \\
    8 & $(7,0)_1, (6,1)_1, (5,2)_1, (4,3)_1$ & 16 & 72 \\
    9 & $(8,0)_1, (7,1)_1, (6,2)_1, (5,3)_1, (4,4)$ & 18 & 90 
    \end{tabular}
    \caption{Showing shell closure for first $9$ shells. Degeneracy follows $d = 2R$ with energy per particle $R\hbar \omega$. The subscript 1 means the set on $n$ can be permuted once, $(x,y)_1 = (x,y),(y,x)$.}\label[tab]{tab:met:ho_shell_numbers}
\end{table}
The interactive $N=2$ case for $\omega = 1$ has been solved analytically, giving a ground state energy of $3$\au \citep{tautTwoElectronsHomogeneous1994}. In addition, \citep{pedersenlohneInitioComputationEnergies2011} have tabulated multiple $\omega$ frequencies with up to $N = 20$ electrons, calculated for a variety of CC truncation and schemes.

\subsubsection{Doubly Magic Nuclei}
Define
\begin{align}
    R
\end{align}