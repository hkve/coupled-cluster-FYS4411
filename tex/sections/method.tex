\subsection{Quantum Mechanical System}

\subsubsection{Helium and Beryllium}
The initial testing during development of the CCD and RCCD implementations was performed using Hydrogen wave functions. As a choice of basis sets, these functions are `physically motivated` in the sense of being solutions to the one body electron case. The well-know relevant quantum numbers determining the form of the spatial wave functions are $n$ as the principal quantum number, with $l$ and $m$ as orbital angular momentum and projection respectively. Due to the spherically symmetric potential, the wave function $\psi_{nlm}$ can be separated in a radial function $R_{nl}(r)$ and a spherical harmonic $\sphericalharmonic{m}{l}$,  

\begin{align}
    \psi_{nlm}(r,\theta,\phi) = R_{nl}(r) \sphericalharmonic{m}{l}(\theta, \phi). \label[eq]{eq:met:hydrogen_wf_definition}
\end{align}
The radial part $R_{nl}(r)$ has the form

\begin{align*}
    R_{nl}(r) &= A_{nl} e^{-r/na_0} \pclosed{\frac{2r}{na_0}}^l \bclosed{\associatedlaguerre{2l+1}{n-l-1}(2r/na_0)} \\
    A_{nl} &= \sqrt{ \pclosed{\frac{2}{na_0}}^3 \frac{(n-l-1)!}{2n(n+l)!} }
\end{align*}
With $L$ as the associated Laguerre polynomials and $a_0$ the Bohr radius. For $l > 0$ the Coulomb interaction integral of \cref{eq:met:hydrogen_wf_definition} can not be easily evaluated due to $\sphericalharmonic{m}{l}$ having a non-trivial $\theta$ and $\phi$ dependence. Therefor for simplicity we only consider $s$ orbital ($l=0$ states), when calculating $\elm{pq}{rs}$. This is briefly sketched in \cref{sec:app:hydrogen_coulomb_integrals}. The one-body term $\hshort{p}{q}$ are diagonal with 

\begin{align}
    h_{nm} = -\frac{Z}{2n^2}\delta_{nm}
\end{align}
We will be interested in calculating the ground state energy for Helium and Beryllium, with two and four electrons respectively. In addition to Hartree-Fock and CCD calculations, a comparison to configuration interaction using singles (CIS) will be performed for both Helium and Beryllium. The results will also be benchmarked against the famous work done by Egil A. Hylleraas \citep{hylleraasNumerischeBerechnung2STerme1930}.

\subsubsection{Two-Dimensional Harmonic Oscillator}
To test the HF and CCD implementations on larger basis sets, the two-dimensional harmonic oscillator was chosen. $N$ electrons are confined in a potential characterized by the oscillation frequency $\omega$, with a repulsive Coulomb term

\begin{align}
    H = \sum_{i=1}^N \pclosed{-\frac{1}{2}\nabla_i^2 + \frac{1}{2}\omega^2 r_i^2} + \sum_{i < j} \frac{1}{r_{ij}} \label[eq]{eq:met:2dho_hamiltonian}
\end{align}
where $r_{ij} = |\vec{r}_i - \vec{r}_j|$. The natural single particle basis for this problem is the solutions to the non-interactive harmonic oscillator case. Characterized by two quantum numbers $n_x$ and $n_y$, the position space wave function in Cartesian coordinates are expressed as 

\begin{align*}
    \psi_{n_x n_y} (x,y) &= A_{n_x n_y} H_{n_{x}} (\sqrt{\omega}x) H_{n_{y}} (\sqrt{\omega}y) e^{-\omega (x^2 + y^2)/2} \\
    A_{n_x n_y} &= \sqrt{\frac{\omega}{\pi 2^{(n_x+n_y)}n_x ! n_y !}}
\end{align*}
The one-body Hamiltonian is diagonal, with the well known energies

\begin{align}
    E_{n_x,n_y} = \hbar \omega (n_x + n_y + 1) \label[eq]{eq:met:ho_sp}
\end{align}
for two dimensions. For the Coulomb integral from \cref{eq:met:2dho_hamiltonian} a more nuanced consideration is in order. The simplest approach is to solve the Coulomb integrals numerically, however even though we just work in two dimensions, the integral will be two-dimensional with a quite complex integrand. For larger basis sets where $L \sim 100$, this approach can not be performed without any clever performance tricks. Luckily this problem has been solved analytically spherical coordinates \citep{anisimovasEnergySpectraFewelectron1998}, giving a much cheaper way to incorporate the Coulomb integrals. The implementation co-authored by Ø. Schøyen has been used \footnote{\url{https://github.com/HyQD/quantum-systems/tree/master}}.

We will consider closed shell systems, that is the particle number $N$ will always correspond to all sets of $(n_x, n_y)$ which gives the same single particle energy from \cref{eq:met:ho_sp} and every energy below this. These shell closures are tabulated in \cref{tab:met:ho_shell_numbers}.

\begin{table}[H]
    \centering
    \begin{tabular}{c|ccc}
    $R$ &  $(n_x,n_y)$ & $d$ & $N$ \\
    \hline
    1 & $(0,0)$ & 2 & 2\\
    2 & $(1,0)_1$ & 4 & 6\\
    3 & $(2,0)_1, (1,1)$ & 6 & 12\\
    4 & $(3,0)_1,(2,1)_1 $ & 8 & 20 \\
    5 & $(4,0)_1, (3,1)_1, (2,2)$ & 10 & 30 \\
    6 & $(5,0)_1, (4,1)_1, (3,2)_1$ & 12 & 42 \\
    7 & $(6,0)_1, (5,1)_1, (4,2)_1,(3,3)$ & 14 & 56 \\
    8 & $(7,0)_1, (6,1)_1, (5,2)_1, (4,3)_1$ & 16 & 72 \\
    9 & $(8,0)_1, (7,1)_1, (6,2)_1, (5,3)_1, (4,4)$ & 18 & 90 
    \end{tabular}
    \caption{Showing shell closure for first $9$ shells. Degeneracy follows $d = 2R$ with energy per particle $R\hbar \omega$. The subscript 1 means the set on $n$ can be permuted ones, $(x,y)_1 = (x,y),(y,x)$.}\label[tab]{tab:met:ho_shell_numbers}
\end{table}

\subsubsection{Doubly Magic Nuclei}
Define
\begin{align}
    R
\end{align}