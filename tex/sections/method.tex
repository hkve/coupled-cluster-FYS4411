\subsection{Quantum Mechanical System}

\subsubsection{Helium and Beryllium}
The initial testing during development of the CCD and RCCD implementations was performed using Hydrogen wave functions. As a choice of basis sets, these functions are `physically motivated` in the sense of being solutions to the one body electron case. The well-know relevant quantum numbers determining the form of the spatial wave functions are $n$ as the principal quantum number, with $l$ and $m$ as orbital angular momentum and projection respectively. Due to the spherically symmetric potential, the wave function $\psi_{nlm}$ can be separated in a radial function $R_{nl}(r)$ and a spherical harmonic $\sphericalharmonic{m}{l}$,  

\begin{align}
    \psi_{nlm}(r,\theta,\phi) = R_{nl}(r) \sphericalharmonic{m}{l}(\theta, \phi). \label[eq]{eq:met:hydrogen_wf_definition}
\end{align}
The radial part $R_{nl}(r)$ has the form

\begin{align*}
    R_{nl}(r) &= A_{nl} e^{-r/na_0} \pclosed{\frac{2r}{na_0}}^l \bclosed{\associatedlaguerre{2l+1}{n-l-1}(2r/na_0)} \\
    A_{nl} &= \sqrt{ \pclosed{\frac{2}{na_0}}^3 \frac{(n-l-1)!}{2n(n+l)!} }
\end{align*}
With $L$ as the associated Laguerre polynomials and $a_0$ the Bohr radius. For $l > 0$ the Coulomb interaction integral of \cref{eq:met:hydrogen_wf_definition} can not be easily evaluated due to $\sphericalharmonic{m}{l}$ having a non-trivial $\theta$ and $\phi$ dependence. Therefor for simplicity we only consider $s$ orbital ($l=0$ states), when calculating $\elm{pq}{rs}$. This is briefly sketched in \cref{sec:app:hydrogen_coulomb_integrals}. The one-body term $\hshort{p}{q}$ are diagonal with 

\begin{align}
    h_{nm} = -\frac{Z}{2n^2}\delta_{nm}
\end{align}
We will be interested in calculating the ground state energy for Helium and Beryllium, with two and four electrons respectively. In addition to Hartree-Fock and CCD calculations, a comparison to configuration interaction using singles (CIS) will be performed for both Helium and Beryllium. The results will also be benchmarked against the famous work done by Egil A. Hylleraas \citep{hylleraasNumerischeBerechnung2STerme1930}.

\subsubsection{Two-Dimensional Harmonic Oscillator}
To test the HF and CCD implementations on larger basis sets, the two-dimensional harmonic oscillator was chosen. $N$ electrons are confined in a potential characterized by the oscillation frequency $\omega$, with a repulsive Coulomb term

\begin{align}
    H = \sum_{i=1}^N \pclosed{-\frac{1}{2}\nabla_i^2 + \frac{1}{2}\omega^2 r_i^2} + \sum_{i < j} \frac{1}{r_{ij}} 
\end{align}
where $r_{ij} = |\vec{r}_i - \vec{r}_j|$. The natural single particle basis for this problem is the solutions to the non-interactive harmonic oscillator case. Characterized by two quantum   

\subsubsection{Doubly Magic Nuclei}
Define
\begin{align}
    R
\end{align}