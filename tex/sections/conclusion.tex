The implementations of HF and CCD, in both restricted and unrestricted schemes were successful. Benchmarking the $s$-orbital hydrogen basis, we found that CCD using a HF basis outperformed CIS for both Helium and Beryllium. The improvement over CI was particularly good for Beryllium, reducing the relative error from $2.08\%$ for CI to $1.05\%$ for CCD using a HF basis. However, most of the improvement was achieved due to the HF basis being particularly good.

For the QD, the $N = 2, \omega = 1.0$ analytical result was reproduced using $12$ shells with a relative error of only $0.196\%$. Here HF produced a relative error of $5.397\%$, which displays that electron correlations is an important factor in the HO system. Increasing the number of particles increased the energy per particle, as expected by the non-extensivity of the electron-electron repulsion. Decreasing the frequency  entailed convergence problems, which was identified as problems with the computational basis since the interaction became more prevalent than the HO potential. For 2 and 6 particles, low frequency solutions were obtained giving critical frequencies of $<0.010$ and $0.0061$ respectively.

The extra considerations with explicitly summing spin out of the equations was expedient, as was found when benchmarking runtime between the different methods. Further work would be to optimize even more, making use of intermediates for the RCCD implementation and matrix storage of elements and amplitudes.