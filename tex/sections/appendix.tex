\subsection{Hydrogen Coulomb integrals}\label[app]{sec:app:hydrogen_coulomb_integrals}
Considering states without orbital angular momentum, we remove the dependence on the two quantum numbers $l$ and $m$, giving

\begin{align*}
    &\psi_{nlm}(r,\theta,\phi) \longrightarrow \psi_n (r,\theta, \phi) \\
    &= \sqrt{\pclosed{\frac{4}{n^5}} } e^{-r/n} \associatedlaguerre{1}{n-1}(2r/n) \sphericalharmonic{0}{0}.
\end{align*}
Where we work in distances of the Bohr radius $r/a_0 \longrightarrow r$. Since the Coulomb integral is over two $\vec{r}_1, \vec{r}_2 \in \mathbb{R}^3$ spaces, we align $\vec{r}_1$ along the $y$-axis and perform the $\vec{r}_2$ integral first. In spherical coordinates, the Coulomb interaction then becomes 

\begin{align}
    \hat{v}(\vec{r}_1, \vec{r}_2) = \frac{Z}{|\vec{r}_1 - \vec{r}_2|} = \frac{Z}{\sqrt{r_1^2 + r_2 ^2 - 2r_1 r_2 \cos \theta_2 }} \label[eq]{eq:app:coulomb_interaction}
\end{align}
With these preparations, the integrals can be solved for all $p,q,r,s$ combinations. The integrals were solved using \verb|SymPy| \citep{10.7717/peerj-cs.103}

\subsection{Detailed Spin Restricted Coupled Cluster Energy}\label[app]{sec:app:ccd_spinrestriction}
We begin by expanding the antisymmetrized terms in the CCD energy expression from \cref{eq:theo:Energy_CCD_explicit}
\begin{align*}
    \DECCD  = \frac{1}{4} \sum_{\substack{ab\\ij}} \elmASshort{ij}{ab}\amplitude{ab}{ij} = \frac{1}{8}\sum_{\substack{ab\\ij}} (\elmshort{ij}{ab} - \elmshort{ij}{ba})(\amplitude{ab}{ij}- \amplitude{ba}{ij})
\end{align*}
The restricted amplitudes are related to the unrestricted by
\begin{align*}
    \amplitude{ab}{ij} = \amplituderestrict{ab}{ij} - \amplituderestrict{ba}{ij}
\end{align*}
Meaning that
\begin{align*}
    \hspace{15px}\amplitude{ab}{ij}-\amplitude{ba}{ij} &= \pclosed{\amplituderestrict{ab}{ij}-\amplituderestrict{ba}{ij}} - \pclosed{\amplituderestrict{ba}{ij}-\amplituderestrict{ab}{ij}} \\
    &= 2\pclosed{\amplituderestrict{ab}{ij}-\amplituderestrict{ba}{ij}}
\end{align*}
we insert the restricted amplitudes and expand 
\begin{align*}
    \DECCD &= \frac{1}{4}\sum_{\substack{ab\\ij}} (\elmshort{ij}{ab} - \elmshort{ij}{ba})(\amplituderestrict{ab}{ij}- \amplituderestrict{ba}{ij}) \\
    &= \frac{1}{4} \sum_{\substack{ab\\ij}} \elmshort{ij}{ab}\amplituderestrict{ab}{ij} + \underbrace{\elmshort{ij}{ba}\amplituderestrict{ba}{ij}}_{C_1} - \elmshort{ij}{ba}\amplituderestrict{ab}{ij} - \underbrace{\elmshort{ij}{ab}\amplituderestrict{ba}{ij}}_{C_2}.
\end{align*}
Then by application of the matrix element \cref{eq:theo:matrix_elements_symmetry} and amplitude \cref{eq:met:amplitude_restricted_symmetry} symmetries, we see from the terms $C_1$ and $C_2$,
\begin{align*}
    C_1 &= \sum_{\substack{ab\\ij}}\elmshort{ij}{ba}\amplituderestrict{ba}{ij} = \sum_{\substack{ab\\ij}}\elmshort{ij}{ba}\amplituderestrict{ab}{ji} = \sum_{\substack{ab\\ij}}\elmshort{ji}{ab}\amplituderestrict{ab}{ji} = \sum_{\substack{ab\\ij}}\elmshort{ij}{ab}\amplituderestrict{ab}{ij} \\
    C_2 &= \sum_{\substack{ab\\ij}} \elmshort{ij}{ab}\amplituderestrict{ba}{ij} = \sum_{\substack{ab\\ij}} \elmshort{ji}{ba}\amplituderestrict{ab}{ji} =  \sum_{\substack{ab\\ij}} \elmshort{ij}{ba}\amplituderestrict{ab}{ij}
\end{align*}
Inserting back into $\DECCD$ we find
\begin{align*}
    \DECCD  &= \frac{1}{2} \sum_{\substack{ab\\ij}}\elmshort{ij}{ab} \amplituderestrict{ab}{ij} - \elmshort{ij}{ba}\amplituderestrict{ij}{ab}
\end{align*}
Now we explicitly sum over the spins $\set{\sigma_A,\sigma_B,\sigma_I,\sigma_J}$ giving
\begin{align*}
    \sum_{\substack{ab\\ij}}\elmshort{ij}{ab} \amplituderestrict{ab}{ij} &= \sum_{\substack{AB\\IJ}}\sum_{\substack{\sigma_A \sigma_B\\\sigma_I \sigma_J}} \elmshort{IJ}{AB}\amplituderestrict{AB}{IJ} \delta_{\sigma_A \sigma_I} \delta_{\sigma_B \sigma_J} \\
    &= \sum_{\substack{AB\\IJ}}\sum_{\substack{\sigma_A \sigma_B}} \elmshort{IJ}{AB}\amplituderestrict{AB}{IJ} = 4 \sum_{\substack{AB\\IJ}} \elmshort{IJ}{AB}\amplituderestrict{AB}{IJ} \\
    \sum_{\substack{ab\\ij}}\elmshort{ij}{ba} \amplituderestrict{ab}{ij} &= \sum_{\substack{AB\\IJ}}\sum_{\substack{\sigma_A \sigma_B\\\sigma_I \sigma_J}} \elmshort{IJ}{BA}\amplituderestrict{AB}{IJ} \delta_{\sigma_B \sigma_I} \delta_{ \sigma_A \sigma_J} \delta_{\sigma_A \sigma_I} \delta_{ \sigma_B \sigma_J} \\
    &= \sum_{\substack{A B\\I J}} \sum_{\substack{\sigma_A \sigma_B}} \elmshort{IJ}{BA}\amplituderestrict{AB}{IJ} \delta_{\sigma_A \sigma_B} \\
    &= 2\sum_{\substack{A B\\I J}}  \elmshort{IJ}{BA}\amplituderestrict{AB}{IJ}
\end{align*}
Giving the total energy expression after spin has been summed out
\begin{align*}
    \DECCD  &= \frac{1}{2}\sum_{\substack{ab\\ij}} \elmshort{ij}{ab} \amplituderestrict{ab}{ij} - \elmshort{ij}{ba} \amplituderestrict{ab}{ij} \\
    &= \sum_{\substack{AB\\IJ}} \pclosed{2 \elmshort{IJ}{AB} - \elmshort{IJ}{BA}} \amplituderestrict{AB}{IJ}
\end{align*}