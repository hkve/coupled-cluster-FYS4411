Quantum dots (QDs) have emerged as one of the most fascinating and promising fields of research in the realm of nanotechnology. Electrons are confined to a small volume, typically measuring a few nanometers in diameter, exhibit unique and extraordinary properties due to their quantum confinement effects. From the discrete energy levels, emission and absorption effects makes them resemble artificial atoms \citep{atomlike}, with tunable properties such as confinement strength. Famously they have been used for the QLED technology \citep{QLED} being in the forefront of screen and television development. In addition to optical properties, use cases in transistors and solar cells are also a hot topic due to their electroical and thermal properties. By entangling multiple dots containing single electrons at low temperatures, applications in quantum computing are also prevalent.

In this work we will investigate two-dimensional QDs confined in a harmonically oscillating (HO) potential. Closed shell systems will be considered, using 2, 6, 12 and 20 electrons in the HO trap with an adjustable frequency. Constructing a state with solution from the single particle HO oscillator does give a rough estimate on ground state energies. However, due to the Coulomb repulsion between electrons, simply considering $N$ non-interacting particles is not sufficient for even qualitative descriptions of the system. Accounting for this two-body interaction is the impediment when considering ground state energy calculations.

To tackle the high computational complexity of many-body quantum mechanical systems, approximate methods are always needed. We will consider the ab initio Coupled cluster (CC) method, where the multielectron state is approximated by excitations created by the exponential of a cluster operator. The exponential ansatz is perturbative in nature and will be truncated to only include cluster operator creating double excitations, named the Coupled-cluster doubles (CCD) approximation. This will be built upon a more common Hartree-Fock (HF) calculation, where the cheaper HF calculation provides a better basis for computations.