\subsection{Helium and Beryllium}
The Helium and Beryllium calculations were presented in \cref{tab:res:helium_and_beryllium}. Comparing with the FCI calculated the values, we see from the relative error \cref{tab:res:helium_and_beryllium_percent_error} that performing a CCD calculation using either the Hydrogen basis or HF basis improved over the reference energy in each scheme. Comparing CI with HF, we see that CI performed better for Helium while HF actually performing substantially better for Beryllium. CCD using the HF basis did beat both CI and HF in both cases, however the margins were not enormous.

The Hylleraas result \citep{hylleraasNumerischeBerechnung2STerme1930} was $-79.005 \text{ eV}$ or $-2.903$ in atomic units is however much more precise. Inclusion of larger parts of the space, such as considering CISD and CCSD would improve results. Small basis sets such as this could serve as a good benchmarking system going forwards with the more complex solvers.

One could of course also expand the space, including more and more $s$-orbitals. There is however a quite limiting basis when only $l=0$ basis states are included. Lifting the $l = 0$ was explored in this work, however difficulties trough  angular integrands arose, such as

\begin{align*}
    \frac{|\sphericalharmonic{m_1}{l_1}(\theta_1,\phi_1)|^2|\sphericalharmonic{m_2}{l_2}(\theta_2,\phi_2)|^2}{\sqrt{r_1^2 + r_2 ^2 - 2r_1 r_2 \cos \theta_2 }}.
\end{align*}
Further investigation into integration techniques concerning spherical harmonics would be required. The algebraic solver \verb|SymPy| did require a lot of `hand holding` in terms of the $s$-integrals, and presumably this would also be the case when considering less trivial angular integrals. Using one of the many Slater-type orbitals (STOs) or Gaussian-type orbitals (GTOs) is commonly the preferred method in the literature, having closed form recursion relation and being well documented. The problem of $l \neq 0$ integrals might be one of the reasons why Hydrogen-like orbitals are not as prevalent, despite being physically motivated for smaller atoms.  

\subsection{Performance}
Considering the time benchmarking from \cref{fig:res:timing}, it was clear that the restricted implementations were a worthwhile task. For $R > 9$, the unrestricted CCD had a quite large gap in performance. This could be due to the $\mathcal{O}(M^4 N^2)$ scaling from \cref{eq:met:term_to_be_optimized}, however another hypothesis is the memory usage. In the naive storage scheme where $\elmASshort{pq}{rs}$ needs to store $L^4$ elements, $R = 10$ would 

Better storage schemes such as matrix representations mapping $ij \rightarrow I$ and $ab \rightarrow A$ could increase computational performance, due to better usage of \verb|BLAS| functionality. This was not done however due to the main performance focus being comparing restricted to unrestricted schemes. In future work, a more systemic consideration of computational complexity would be desirable, pushing the limits on larger basis sets.



\subsection{Harmonic Oscillator}
Comparing with the analytical result for $\omega = 1.0$ and two particles, the relative error using a HO and HF basis have 2.997\% and 0.196\% respectively with $R=12$ basis states. This difference is quite more substantial compared with the Helium and Beryllium results. This indicates that the mean field approximation of the average electron-electron interaction is somewhat representative of the system. On the other hand, a simple HF calculation produced a relative error of 5.397\%, meaning that electron-electron correlations also play an important role when one wishes to describe the system more accurately. Comparing with \citep{pedersenlohneInitioComputationEnergies2011}, the same potential using $R=12$ shells as basis states yielded a relative error of $0.0218\%$. This indicates that the inclusion of singles excitations also plays an important role to achieve high precision results.
\subsubsection{Increasing the Number of Particles}
From the detailed $\omega = 1.0$ for 2, 6, 12 and 20 particles \cref{tab:res:ho_omega1} the energy per particle increases. For $N=2$ we have $E/N = 1.502990$, increasing to 3.367877 for $N=6$ and 5.487481 for $N=12$, while the $N = 20$ case tops it off with $E/N=7.811913$. Comparing with the non-interacting solutions of 1.00, 1.67, 2.34 and 3.00, we move from having $\sim 150\%$ of the non-interacting solution for 2 particles to $\sim 260\%$ for 20 particles. It is clear that adding more interactive particles in the trap increases the energy due to electron-electron repulsion.

Convergence using the HO basis was difficult for 12 and 20 particles. This can be explained by the true eigenstates of the interacting system deviating from the HO solution when the interaction plays a larger role. By making linear combinations of HO basis states and accounting for some electron-electron interaction by a mean field approximation, more accurate single particle eigenstates are created and thus convergence was achieved using the HF basis.

\subsubsection{Turing the Frequency Down}
The $\omega = 0.5$ results from \cref{tab:res:ho_omega0.5} yielded an energy per particle of 0.831762, 1.970973 and 3.273831 for 2, 6 and 12 particles respectively. Comparing with the non-interacting case for 0.5, we contain $\sim 160\%,\sim 240\%$ and $\sim 280\%$ of the non-interacting solution for 2, 6 and 12 particles. Thus decreasing the frequency makes the electron-electron repulsion play a bigger role than the HO trap. Making the HF basis converge is an easier job than the HO basis for the same reasons as for the $\omega = 1.0$ case, just more prevalent due to the interaction playing a bigger role. Additionally, achieving convergence for 20 particles using the HF basis also presented some problems. There are two reasons for this.

Firstly CC is perturbative in nature, meaning that when the non-interacting single particle energies moves closer and closer (since the frequency decreases), we move towards an almost degenerate state. As seen in \cref{eq:met:amplitudes_update_rule}, we divide by the diagonal Fock matrix, where the main contribution is the single particle energies. When  $\epsilon_a + \epsilon_b - \epsilon_i - \epsilon_j \rightarrow 0$, the new amplitude contribution will be inflated. This can be counteracted by turning up the mixing parameter $p$, but this might yield slow or no convergence of the amplitudes.

Secondly for low frequencies, the electron-electron repulsion be so large that the HO trap is not strong enough to contain the particles, such that we actually have an unbound system. For an unbound system, the HO solutions are not the correct basis states despite showing much of the same sinusoidal behavior as the free particle.   

These effects are even more prevalent for the $\omega = 0.1$ system \cref{tab:res:ho_omega0.1}, where convergence was only possible for 2 and 6 particles. Here the system is suspected to be unbound when filling the $R=3$ shell completely. Studying the low frequency domain in more detail for $R=12$ viritual states we found from \cref{fig:res:varying_omega} and \cref{tab:res:omega_critical} that non-convergence did not occur when pushing $\omega$ down to 0.01 for 2 particles. Increasing to $N = 6$ particles, the lowest frequency was found to be $\omega_c = 0.0621$. This could presumably be pushed slightly lower with a more detailed mixing parameter tuning. Here the calculated energy was over $400\%$ of the non-interactive system, where the validity of the HO basis functions are quite remarkable. This was of course calculated with the HF basis, displaying the power of letting constructive and destructive interference play a part in making new single particle states. For 12 and 20 particles however the reduce frequency did not produce any major improvements over the $\omega = 0.5$ results of \cref{tab:res:ho_omega0.5_N12} and $\omega = 1.0$ \cref{tab:res:ho_omega1_N20}, giving critical frequencies of $\omega_c = 0.4747$ and $0.9878$ respectively.  